\documentclass[a4paper,draft]{amsproc}
\title{\textbf{Powerset}}

\date{}
\begin{document}

\theoremstyle{plain}
 \newtheorem{ftltheorem}{Theorem}[section]
 \newtheorem{theorem}{Theorem}[section]
 \newtheorem{ftlproposition}{Proposition}[section]
 \newtheorem{ftllemma}{Lemma}[section]
 \newtheorem{ftlcorollary}{Corollary}[section]
\theoremstyle{definition}
 \newtheorem{example}{Example}[section]
 \newtheorem{ftldefinition}{Definition}[section]
\theoremstyle{remark}
 \newtheorem{remark}{Remark}[section]
 \newtheorem{notation}{Notation}[section]
\theoremstyle{axiom}
 \newtheorem{ftlaxiom}{Axiom}[section]
 \numberwithin{equation}{section}

\newenvironment{parser}{}{}

\maketitle 
\section{First and only section}

\begin{theorem}
The following holds: $1 + 1 = 2$. This is not parsed by forthel.
\end{theorem}

\begin{parser}
[synonym subset/-s]
\end{parser}

\begin{parser}
[synonym surject/-s]
\end{parser}

\begin{notation}
Let M denote a set.
\end{notation}

\begin{notation}
Let f denote a function.
\end{notation}

\begin{notation}
Let the value of f at x stand for f[x].
\end{notation}

\begin{notation}
Let f is defined on M stand for Dom(f) = M.
\end{notation}

\begin{notation}
Let the domain of f stand for Dom(f).
\end{notation}


\begin{ftlaxiom}
The value of f at any element of the domain of f is a set.
\end{ftlaxiom}

\begin{ftldefinition}
A subset of M is a set N such that every element of N is an element of M.
\end{ftldefinition}

\begin{ftldefinition}
The powerset of M is the set of subsets of M.
\end{ftldefinition}

\begin{ftldefinition}
f surjects onto M iff every element of M is equal to the value of f at some element of the domain of f.
\end{ftldefinition}

\begin{ftlproposition}
No function that is defined on M surjects onto the powerset of M.
Proof by contradiction.
Assume the contrary. Take a function f that is defined on M and surjects onto the powerset of M.
Define N = { x in M | x is not an element of f[x] }.
Then N is not equal to the value of f at any element of M.
Contradiction. qed.
\end{ftlproposition}

\begin{ftltheorem}
The value of f at any element of the domain of f is a set.
\end{ftltheorem}

\end{document}