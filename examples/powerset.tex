\documentclass[a4paper,draft]{amsproc}
\title{\textbf{Powerset}}

\date{}
\begin{document}


\newenvironment{parser}{}{}
\newenvironment{pretype}{}{}
\newenvironment{macro}{}{}
\newenvironment{axiom}{}{}
\newenvironment{definition}{}{}
\newenvironment{proposition}{}{}
\newenvironment{theorem}{}{}

\maketitle 
\section{First and only section}

\begin{parser}
[synonym subset/-s]
\end{parser}

\begin{parser}
[synonym surject/-s]
\end{parser}

\begin{pretype}
Let M denote a set.
\end{pretype}

\begin{pretype}
Let f denote a function.
\end{pretype}

\begin{macro}
Let the value of f at x stand for f[x].
\end{macro}

\begin{macro}
Let f is defined on M stand for Dom(f) = M.
\end{macro}

\begin{macro}
Let the domain of f stand for Dom(f).
\end{macro}


\begin{axiom}
The value of f at any element of the domain of f is a set.
\end{axiom}

\begin{definition}
A subset of M is a set N such that every element of N is an element of M.
\end{definition}

\begin{definition}
The powerset of M is the set of subsets of M.
\end{definition}

\begin{definition}
f surjects onto M iff every element of M is equal to the value of f at some element of the domain of f.
\end{definition}

\begin{proposition}
No function that is defined on M surjects onto the powerset of M.
Proof by contradiction.
Assume the contrary. Take a function f that is defined on M and surjects onto the powerset of M.
Define N = { x in M | x is not an element of f[x] }.
Then N is not equal to the value of f at any element of M.
Contradiction. qed.
\end{proposition}

\begin{theorem}
The value of f at any element of the domain of f is a set.
\end{theorem}

\end{document}