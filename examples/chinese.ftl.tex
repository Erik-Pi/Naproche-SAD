\documentclass{article}

\usepackage[utf8]{inputenc}
\usepackage[english]{babel}
\usepackage{../lib/tex/forthel}

\title{Chinese remainder theorem}
\author{Andrei Paskevich}
\date{}

\begin{document}
  \pagenumbering{gobble}

  \maketitle

  \begin{forthel}
    [synonym element/-s]

    \begin{signature}[ElmSort]
      An element is a notion.
    \end{signature}

    Let $a,b,c,x,y,z,u,v,w$ denote elements.

    \begin{axiom}
      $a$ is setsized.
    \end{axiom}

    \begin{signature}[SortsC]
      $0$ is an element.
    \end{signature}

    \begin{signature}[SortsC]
      $1$ is an element.
    \end{signature}

    \begin{signature}[Sortsu]
      $-x$ is an element.
    \end{signature}

    \begin{signature}[SortsB]
      $x + y$ is an element.
    \end{signature}

    \begin{signature}[SortsB]
      $x * y$ is an element.
    \end{signature}

    Let $x$ is nonzero stand for $x != 0$.
    Let $x - y$ stand for $x + -y$.

    \begin{axiom}[AddComm]
      $x + y = y + x$.
    \end{axiom}

    \begin{axiom}[AddAsso]
      $(x + y) + z = x + (y + z)$.
    \end{axiom}

    \begin{axiom}[AddBubble]
      $x + (y + z) = y + (x + z)$.
    \end{axiom}

    \begin{axiom}[AddZero]
      $x + 0 = x = 0 + x$.
    \end{axiom}

    \begin{axiom}[AddInvr]
      $x + -x = 0 = -x + x$.
    \end{axiom}

    \begin{axiom}[MulComm]
      $x * y = y * x$.
    \end{axiom}

    \begin{axiom}[MulAsso]
      $(x * y) * z = x * (y * z)$.
    \end{axiom}

    \begin{axiom}[MulBubble]
      $x * (y * z) = y * (x * z)$.
    \end{axiom}

    \begin{axiom}[MulUnit]
      $x * 1 = x = 1 * x$.
    \end{axiom}

    \begin{axiom}[AMDistr1]
      $x * (y + z) = (x * y) + (x * z)$.
    \end{axiom}

    \begin{axiom}[AMDistr2]
      $(y + z) * x = (y * x) + (z * x)$.
    \end{axiom}

    \begin{axiom}[MulMnOne]
      $1 * x = -x = x * -1$.
    \end{axiom}

    \begin{lemma}[MulZero]
      $x * 0 = 0 = 0 * x$.
    \end{lemma}
    \begin{proof}
      Let us show that $x * 0 = 0$.
        $x * 0 .= x * (0 + 0)$ (by AddZero) $.= (x * 0) + (x * 0)$ (by AMDistr1).
      end.

      Let us show that $0 * x = 0$.
        $0 * x .= (0 + 0) * x$ (by AddZero) $.= (0 * x) + (0 * x)$ (by AMDistr2).
      end.
    \end{proof}

    \begin{axiom}[Cancel]
      $x != 0 /\ y != 0 => x * y != 0$.
    \end{axiom}

    \begin{axiom}[UnNeZr]
      $1 != 0$.
    \end{axiom}

    [synonym set/-s] [synonym belong/-s]

    Let $X,Y,Z,U,V,W$ denote sets.

    \begin{axiom}
      $X$ is setsized.
    \end{axiom}

    \begin{axiom}
      Every element of $X$ is an element.
    \end{axiom}

    Let $x << W$ denote $x$ is an element of $W$.
    Let $x$ belongs to $W$ denote $x$ is an element of $W$.

    \begin{axiom}[SetEq]
      If every element of $X$ belongs to $Y$ and every element of $Y$ belongs to $X$ then $X = Y$.
    \end{axiom}

    \begin{definition}[DefSum]
      $X +' Y$ is a set such that for every element $z$ ($z << X +' Y$) iff there exist $x << X, y << Y$ such that $z = x + y$.
    \end{definition}

    \begin{definition}[DefSInt]
      $X ** Y$ is a set such that for every element $z$ ($z << X ** Y$) iff $z << X$ and $z << Y$.
    \end{definition}

    [synonym ideal/-s]

    \begin{definition}[DefIdeal]
      An ideal is a set $X$ such that for every $x << X$
        $forall y << X (x + y) << X$ and
        $forall z (z * x) << X$.
    \end{definition}

    Let $I,J$ denote ideals.

    \begin{lemma}[IdeSum]
      $I +' J$ is an ideal.
    \end{lemma}
    \begin{proof}
      Let $x$ belong to ($I +' J$).

      $forall y << (I +' J) (x + y) << (I +' J)$.
      proof.
        Let $y << (I +' J)$.

        (1) Take $k << I$ and $l << J$ such that $x = k + l$.

        (2) Take $m << I$ and $n << J$ such that $y = m + n$.

        $k + m$ belongs to $I$ and $l + n$ belongs to $J$. $x + y .= (k + m) + (l + n)$ (by 1, 2, AddComm,AddAsso,AddBubble). Therefore the thesis.
      end.

      For every element $z$ $(z * x) << (I +' J)$.
      proof.
        Let $z$ be an element.

        (1) Take $k << I$ and $l << J$ such that $x = k + l$.

        $z * k$ belongs to $I$ and $z * l$ belongs to $J$. $z * x .= (z * k) + (z * l)$ (by AMDistr1, 1). Therefore the thesis.
      end.
    \end{proof}


    \begin{lemma}[IdeInt]
      $I ** J$ is an ideal (by DefIdeal).
    \end{lemma}
    \begin{proof}
      Let x belong to $I ** J$. $\forall y << (I ** J) (x + y) << (I ** J)$. For every element $z$ $(z * x) << (I** J)$.
    \end{proof}

    \begin{definition}[DefMod]
      $x = y (mod I)$ iff $x - y << I$.
    \end{definition}

    \begin{theorem}[ChineseRemainder]
      Suppose that every element belongs to $I +' J$. Let $x, y$ be elements. There exists an element $w$ such that $w = x (mod I)$ and $w = y (mod J)$.
    \end{theorem}
    \begin{proof}
      Take $a << I$ and $b << J$ such that $a + b = 1$ (by DefSum).

      (1) Take $w = (y * a) + (x * b)$.
      Let us show that $w = x (mod I)$ and $w = y (mod J)$.
        $w - x$ belongs to $I$.
        proof.
          $w - x = (y * a) + ((x * b) - x)$. $x * (b - 1)$ belongs to $I$. $x * (b - 1) = (x * b) - x$.
        end.

        $w - y$ belongs to $J$.
        proof.
          $w - y = (x * b) + ((y * a) - y)$. $y * (a - 1)$ belongs to $J$. $y * (a - 1) = (y * a) - y$.
        end.
      end.
    \end{proof}


    [synonym number/-s]

    \begin{signature}[NatSort]
      A natural number is a notion.
    \end{signature}

    \begin{axiom}
      Let $n$ be a natural number. Then n is setsized.
    \end{axiom}

    \begin{signature}[EucSort]
      Let $x$ be a nonzero element. $|x|$ is a natural number.
    \end{signature}

    \begin{axiom}[Division]
      Let $x,y$ be elements and $y != 0$. There exist elements $q,r$ such that $x = (q * y) + r$ and $(r != 0 => |r| -<- |y|)$.
    \end{axiom}


    [synonym divisor/-s] [synonym divide/-s]

    \begin{definition}[DefDiv]
      $x$ divides $y$ iff for some $z$ $(x * z = y)$.
    \end{definition}

    Let $x | y$ stand for $x$ divides $y$.
    Let $x$ is divided by $y$ stand for $y | x$.

    \begin{definition}[DefDvs]
      A divisor of $x$ is an element that divides $x$.
    \end{definition}

    \begin{definition}[DefGCD]
      A gcd of $x$ and $y$ is a common divisor $c$ of $x$ and $y$ such that any common divisor of $x$ and $y$ divides $c$.
    \end{definition}

    \begin{definition}[DefRel]
      $x,y$ are relatively prime iff $1$ is a gcd of $x$ and $y$.
    \end{definition}


    \begin{definition}[DefPrIdeal]
      $<c>$ is a set such that for every $z$ $z$ is an element of $<c>$ iff there exists an element $x$ such that $z = c * x$.
    \end{definition}

    \begin{lemma}[PrIdeal]
      $<c>$ is an ideal.
    \end{lemma}
    \begin{proof}
      Let $x$ belong to $<c>$.

      $forall y << <c> x + y << <c>$.
      proof.
        Let $y << <c>$.

        (1) Take an element $u$ such that $c * u = x$.

        (2) Take an element $v$ such that $c * v = y$.

        $x + y .= c * (u + v)$ (by 1, 2, AMDistr1). Therefore the thesis.
      end.

    $forall z z * x << <c>$.
    proof.
      Let $z$ be an element.

      (1) Take an element $u$ such that $c * u = x$.

      $z * x .= c * (u * z)$ (by 1,MulComm,MulAsso, MulBubble). Therefore the thesis.
    end.
  \end{proof}

  \begin{theorem}[GCDin]
    Let $a,b$ be elements. Assume that $a$ is nonzero or $b$ is nonzero. Let $c$ be a gcd of $a$ and $b$. Then $c$ belongs to $<a> +' <b>$.
  \end{theorem}
  \begin{proof}
    Take an ideal $I$ equal to $<a> +' <b>$. We have $0,a << <a>$ and $0,b << <b>$ (by MulZero, MulUnit). Hence there exists a nonzero element of $<a> +' <b>$. Indeed $a << <a> +' <b>$ and $b << <a> +' <b>$ (by AddZero).

    Take a nonzero $u << I$ such that for no nonzero $v << I$ $(|v| -<- |u|)$.
    Proof.

      We can show by induction on $|w|$ that for every nonzero $w << I$ there
        exists nonzero $u << I$ such that for no nonzero $v << I$ $(|v| -<- |u|)$.
      Obvious.
    qed.

    $u$ is a common divisor of $a$ and $b$.
    proof by contradiction.
      Assume the contrary.

      For some elements $x,y$ $u = (a * x) + (b * y)$.
      proof.
        Take $k << <a>$ and $l << <b>$ such that $u = k + l$. Take elements $x,y$ such that ($k = a * x$ and $l = b * y$). Hence the thesis.
      end.

      Case $u$ does not divide $a$.
        Take elements $q,r$ such that $a = (q * u) + r$ and ($r = 0 \/ |r| -<- |u|$) (by Division). $r$ is nonzero. $- (q * u)$ belongs to $I$. $a$ belongs to $I$ (by AddZero). $r = - (q * u) + a$. Hence $r$ belongs to $I$ (by DefIdeal).
      end.

      Case $u$ does not divide $b$.
        Take elements $q,r$ such that $b = (q * u) + r$ and ($r = 0 \/ |r| -<- |u|$) (by Division). $r$ is nonzero. $- (q * u)$ belongs to $I$. $b$ belongs to $I$ (by AddZero). $r = - (q * u) + b$. Hence $r$ belongs to $I$ (by DefIdeal).
      end.
    end.

    Hence $u$ divides $c$.

    Hence the thesis.
    proof.
      Take an element $z$ such that $c = z * u$. Then $c << I$ (by DefIdeal).
    end.
  \end{proof}
\end{forthel}

\end{document}
