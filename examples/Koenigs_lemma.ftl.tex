\documentclass{article}

\usepackage[utf8]{inputenc}
\usepackage[english]{babel}
\usepackage{../lib/tex/forthel}

\title{König's Theorem}
\author{Steffen Frerix}
\date{2018}

\begin{document}
  \pagenumbering{gobble}

  \maketitle

  \begin{forthel}
    [synonym cardinal/-s]
    [synonym sequence/-s]
    [read ZFC.ftl]

    Let the domain of $f$ stand for $Dom(f)$.

    Let $M,N$ denote a set.

    Let $f$ denote a function.
    \begin{axiom}
      $f$ is setsized.
    \end{axiom}
  \end{forthel}


  \section*{Some preliminary set theory}

  \begin{forthel}
    \begin{definition}
      A subset of $M$ is a set $N$ such that every element of $N$ is an element of $M$.
    \end{definition}

    \begin{definition}
      $M \ N = {x "in" M | x "is not an element of" N}$.
    \end{definition}

    \begin{definition}
      Assume $M$ is a subset of the domain of $f$. $f^[M] = {f[x] | x "is an element of" M}$.
    \end{definition}

    \begin{axiom}
      Assume $M$ is a subset of the domain of $f$. Then $f^[M]$ is a set.
    \end{axiom}

    Let the image of $f$ stand for $f^[Dom(f)]$.
  \end{forthel}


  \section*{Cardinals and Cardinality}

  \begin{forthel}
    \begin{signature}
      A cardinal is a set.
    \end{signature}

    Let $A,B,C$ denote cardinals.

    \begin{signature}
      $A < B$ is an atom.
    \end{signature}

    Let $A =< B$ stand for $A = B$ or $A < B$.

    \begin{axiom}
      $A < B < C => A < C$.
    \end{axiom}

    \begin{axiom}
      $A < B$ or $B < A$ or $B = A$.
    \end{axiom}

    Let $A$ is less than $B$ stand for $A < B$.

    \begin{signature}
      The cardinality of $M$ is a cardinal.
    \end{signature}

    Let $card(M)$ denote the cardinality of $M$.

    \begin{axiom}[ImageCard]
      Assume $M$ is a subset of $Dom(f)$. $card(f^[M]) =< card(M)$.
    \end{axiom}

    \begin{axiom}
      Assume the cardinality of $N$ is less than the cardinality of $M$. Then $M \ N$ has an element.
    \end{axiom}

    \begin{axiom}[SurjExi]
      Assume $card(M) =< card(N)$. Assume $M$ has an element. There exists a
      function $f$ such that $N$ is the domain of $f$ and $M$ is the image of $f$.
    \end{axiom}

    \begin{axiom}
      $card(A) = A$.
    \end{axiom}
  \end{forthel}


  \section*{Product and Sum of cardinals}

  \begin{forthel}
    Let $D$ denote a set.

    \begin{definition}
      A sequence of cardinals on $D$ is a function $f$ such that $Dom(f) = D$ and $f[x]$ is a cardinal for every element $x$ of $D$.
    \end{definition}

    \begin{signature}
      Let $kappa$ be a sequence of cardinals on $D$. $SumSet(kappa,D)$ is a set.
    \end{signature}

    \begin{axiom}[SumDef]
      Let $kappa$ be a sequence of cardinals on $D$. $SumSet(kappa,D) = {(n,i) | i "is an element of" D "and" n "is an element of" kappa[i]}$.
    \end{axiom}

    \begin{axiom}
      Let $kappa$ be a sequence of cardinals on $D$. Then $SumSet(kappa,D)$ is a set.
    \end{axiom}

    \begin{definition}
      Let $kappa$ be a sequence of cardinals on $D$. $Sum(kappa,D) = card(SumSet(kappa,D))$.
    \end{definition}

    \begin{signature}
      Let $kappa$ be a sequence of cardinals on $D$. $ProdSet(kappa,D)$ is a set.
    \end{signature}

    \begin{axiom}[ProdDef]
      Let $kappa$ be a sequence of cardinals on $D$. $ProdSet(kappa,D) = {"function" f | Dom(f) = D /\ (f[i] "is an element of" kappa[i] "for every
      element" i "of" D)}$.
    \end{axiom}

    \begin{axiom}
      Let $kappa$ be a sequence of cardinals on $D$. Then $ProdSet(kappa,D)$ is a set.
    \end{axiom}

    \begin{definition}
      Let $kappa$ be a sequence of cardinals on $D$. $Prod(kappa,D) = card(ProdSet(kappa,D))$.
    \end{definition}


    \begin{lemma}[Choice]
      Let $lambda$ be a sequence of cardinals on $D$. Assume that $lambda[i]$ has an element for every element $i$ of $D$. $ProdSet(lambda, D)$ has an element.
    \end{lemma}
    \begin{proof}
      Define $f[i] =$ Choose an element $v$ of $lambda[i]$ in $v$ for $i$ in $D$. Then $f$ is an element of $ProdSet(lambda,D)$.
    \end{proof}
  \end{forthel}


  \section*{Koenig's theorem}

  \begin{forthel}
    \begin{theorem}[Koenig]
      Let $kappa, lambda$ be sequences of cardinals on $D$. Assume that for every element $i$ of $D$ $kappa[i] < lambda[i]$. Then $Sum(kappa,D) < Prod(lambda,D)$.
    \end{theorem}
    \begin{proof}
      Proof by contradiction. Assume the contrary. Then $Prod(lambda,D) =<
      Sum(kappa,D)$. Take a function $G$ such that $SumSet(kappa,D)$ is the domain of $G$ and $ProdSet(lambda,D)$ is the image of $G$. Indeed $ProdSet(lambda, D)$ has an element. Indeed lambda and D have an element. Define $Diag[i] = {G[(n,i)][i] | n "is an element of" kappa[i]}$ for $i$ in $D$.

      For every element $f$ of $ProdSet(lambda, D)$ for every element $i$ of $D$ $f[i]$ is an element of $lambda[i]$. For every element $i$ of $D$ $lambda[i]$ is a set. For every element $i$ of $D$ for every element $d$ of $Diag[i]$ $d$ is an element of $lambda[i]$. For every element $i$ of $D$ $Diag[i]$ is a set.
      
      For every element $i$ of $D$ $card(Diag[i]) < lambda[i]$.
      proof.
        Let $i$ be an element of $D$. Define $F[n] = G[(n,i)][i]$ for $n$ in $kappa[i]$. Then $F^[kappa[i]] = Diag[i]$.
      end.

      Define $f[i] =$ Choose an element $v$ of $lambda[i] \ Diag[i]$ in $v$ for $i$ in $D$. Then $f$ is an element of $ProdSet(lambda,D)$. Take an element $j$ of $D$ and an element $m$ of $kappa[j]$ such that $G[(m,j)] = f$. $G[(m,j)][j]$ is an element of $Diag[j]$ and $f[j]$ is not an element of $Diag[j]$. Contradiction.
    \end{proof}
  \end{forthel}

\end{document}
