\documentclass{article}

\usepackage[utf8]{inputenc}
\usepackage[english]{babel}
\usepackage{../lib/tex/forthel}

\newcommand{\Cong}[4]{#1 #2 \equiv #3 #4}
%\newcommand{\Betw}[3]{#1 {\sim} #2 {\sim} #3}
\newcommand{\Betw}[3]{#1 #2 #3}
\newcommand{\NotBetw}[3]{\lnot #1 #2 #3}
\newcommand{\Col}[3]{\operatorname{Col}(#1, #2, #3)}

\newcommand{\OFS}[8]{\operatorname{OFS}
\left(\begin{smallmatrix}%
#1 & #2 & #3 & #4 \\
#5 & #6 & #7 & #8
\end{smallmatrix}\right)%
}


\title{Tarski's axioms for Euclidian geometry}
\author{Adrian De Lon, Daniel Kollert}
\date{2020}

\begin{document}
  \pagenumbering{gobble}

  \maketitle

  This is a formalization of the beginnings of Tarskian geometry,
  mainly following the outline of \textit{Metamathematische Methoden in der Geometrie} by
  Schwabhäuser, Szmielew and Tarski.
  We refer to this book as \textit{SST} in later comments.


  \section{The language of Tarskian geometry}

  The only objects under consideration are \textit{points}.
  They are subject to two primitive relations:
  the quaternary \textit{congruence} $\Cong{(-)}{(-)}{(-)}{(-)}$
  and the ternary \textit{betweenness} $\Betw{(-)}{(-)}{(-)}$.
  Intuitively, congruence expresses that the distance between the first two points is equal to the distance of the last two points, and betweenness expresses that
  the second point lies on the between the other two on a shared line.


  \begin{forthel}
    [synonym point/-s]

    \begin{signature}
      A point is a notion.
    \end{signature}

    Let $x, y, z, u, v, w, p, q, r, g, h$ denote points.


    \begin{signature}
      $\Cong{x}{y}{v}{w}$ is an atom.
    \end{signature}

    Let $x$ and $y$ are congruent to $v$ and $w$ stand for $\Cong{x}{y}{v}{w}$.
    Let $x-y : v-w$ stand for $x$ and $y$ are congruent to $v$ and $w$.

    \begin{signature}
      $\Betw{x}{u}{y}$ is an atom.
    \end{signature}

    Let $u$ is between $x$ and $y$ stand for $\Betw{x}{u}{y}$.
    Let $x-u-y$ stand for $u$ is between $x$ and $y$.
    Let $\NotBetw{x}{u}{y}$ stand for $u$ is not between $x$ and $y$.


    \begin{definition}
      $\Col{x}{y}{z}$ iff $\Betw{x}{y}{z}$ or $\Betw{y}{z}{x}$ or $\Betw{z}{x}{y}$.
    \end{definition}

    Let $p$ is colinear with $x$ and $y$ stand for $\Col{p}{x}{y}$.
  \end{forthel}




  \paragraph{Reflexivity of congruence}

  \begin{forthel}
    \begin{axiom}[A1]
      We have $\Cong{x}{y}{y}{x}$.
    \end{axiom}
  \end{forthel}


  \paragraph{Transitivity of congruence}

  \begin{forthel}
    \begin{axiom}[A2]
      If $\Cong{x}{y}{z}{u}$ and $\Cong{x}{y}{v}{w}$ then $\Cong{z}{u}{v}{w}$.
    \end{axiom}
  \end{forthel}


  \paragraph{Identity of congruence}

  \begin{forthel}
    \begin{axiom}[A3]
      If $\Cong{x}{y}{z}{z}$ then $x = y$.
    \end{axiom}
  \end{forthel}


  \paragraph{Segment construction} allows us to extend a segment $xy$
  by a length specified by another segment $pq$.

  \begin{forthel}
    \begin{axiom}[A4]
      There exists a point $z$ such that $\Betw{x}{y}{z}$ and $\Cong{y}{z}{p}{q}$.
    \end{axiom}
  \end{forthel}


  \paragraph{Five segments}
  We say that the points $x,y,z,r,u,v,w,p$ are in an \textit{outer five-segment-configuration}
  whenever $\OFS{x}{y}{z}{r}{u}{v}{w}{p}$.

  %This is definition 2.10 in \textit{SST}.
  \begin{forthel}
    \begin{definition}[OFS]
      $\OFS{x}{y}{z}{r}{u}{v}{w}{p}$ iff $x-y-z$ and $u-v-w$ and $x-y : u-v$ and $y-z : v-w$ and $x-r : u-p$ and $y-r : v-p$.
    \end{definition}
  \end{forthel}

  Using the concept of an outer five-segment-configuration we can state axiom A5 in a more concise form.

  \begin{forthel}
  \begin{axiom}[A5]
    If $\OFS{x}{y}{z}{r}{u}{v}{w}{p}$ and $x \neq y$ then $z-r : w-p$.
  \end{axiom}
  \end{forthel}


  \paragraph{Identity of betweenness}

  \begin{forthel}
    \begin{axiom}[A6]
      If $y$ is between $x$ and $x$ then $x = y$.
    \end{axiom}
  \end{forthel}


  \paragraph{Inner Pasch}

  \begin{forthel}
    \begin{axiom}[A7]
      If $x-u-z$ and $y-v-z$ then there exists a point $w$ such that $u-w-y$ and $v-w-x$.
    \end{axiom}
  \end{forthel}


  \paragraph{Lower dimension}

  \begin{forthel}
    \begin{axiom}[A8]
      There exist points $a,b,c$ such that
      $\NotBetw{a}{b}{c}$ and $\NotBetw{b}{c}{a}$ and $\NotBetw{c}{a}{b}$.
    \end{axiom}
  \end{forthel}


  \paragraph{Upper dimension}

  \begin{forthel}
    \begin{axiom}[A9]
      If $x-u : x-v$ and $y-u : y-v$ and $z-u : z-v$ and $u \neq v$ then $x-y-z$ and $y-z-x$ and $z-x-y$.
    \end{axiom}
  \end{forthel}


  \paragraph{Euclid}

  \begin{forthel}
    \begin{axiom}[A10]
      If $x-r-v$ and $y-r-z$ and $x \neq r$ then there exist points $s,t$ such that $x-y-s$ and $x-z-t$ and $s-v-t$.
    \end{axiom}
  \end{forthel}


  \paragraph{Circle continuity axiom} This axiom is equivalent to the the following statement: A line that has a point within a circle intersects that circle (i.e. shares a point with that circle.).

  \begin{forthel}
    \begin{axiom}[CA]
      Assume $z-p-q$ and $z-p-r$ and $z-x : z-p$ and $z-y : z-r$. Then there exists $w$ such that $z-w : z-q$ and $x-w-y$.
    \end{axiom}
  \end{forthel}







  \paragraph{Reflexivity of congruence}

  \begin{forthel}
    \begin{lemma}[L2o1]
      $x-y : x-y$.
    \end{lemma}
  \end{forthel}


  \paragraph{Symmetry of congruence}

  \begin{forthel}
    \begin{lemma}[L2o2]
      Assume $x-y : v-w$. Then $v-w : x-y$.
    \end{lemma}
  \end{forthel}


  \paragraph{Transitivity of congruence}

  \begin{forthel}
    \begin{lemma}[L2o3]
      Assume $x-y : v-w$ and $v-w : p-q$. Then $x-y : p-q$.
    \end{lemma}
  \end{forthel}


  \paragraph{Congruence is independent of the order of the pairs}

  \begin{forthel}
    \begin{lemma}[L2o4]
      Assume $x-y : v-w$. Then $y-x : v-w$.
    \end{lemma}

    \begin{lemma}[L2o5]
      Assume $x-y : v-w$. Then $x-y : w-v$.
    \end{lemma}
  \end{forthel}


  \paragraph{Zero-length segments are congruent}

  \begin{forthel}
    \begin{lemma}[L2o8]
      $x-x : y-y$.
    \end{lemma}
  \end{forthel}


  \paragraph{Concatenation of segments}

  \begin{forthel}
    \begin{lemma} % Satz 2.11
      Assume $\Betw{x}{y}{z}$ and $\Betw{r}{v}{w}$.
      Assume $\Cong{x}{y}{r}{v}$ and $\Cong{y}{z}{v}{w}$.
      Then $\Cong{x}{z}{r}{w}$.
    \end{lemma}
    \begin{proof}
      We have $\OFS{x}{y}{z}{x}{r}{v}{w}{r}$. % By previous results and assumption.
      If $x = y$ then $r = v$.                % Axiom A3 gives this implication.
      If $x \neq y$ then $\Cong{x}{z}{r}{w}$. % Axiom A5 completes the proof.
    \end{proof}
  \end{forthel}


  \paragraph{Uniqueness for axiom A4}

  \begin{forthel}
    \begin{lemma} % Satz 2.12
      Assume $q \neq x$.
      Suppose $q-x-y$ and $x-y : v-w$ and $q-x-z$ and $x-z : v-w$.
      Then $y = z$.
    \end{lemma}
    \begin{proof}
      We have $\Cong{q}{y}{q}{z}$.
      Thus $\Cong{x}{y}{x}{z}$.
      Thus $\OFS{q}{x}{y}{y}{q}{x}{z}{z}$.
      Therefore $\Cong{y}{y}{z}{z}$.
    \end{proof}
  \end{forthel}


  \paragraph{Right-betweenness}

  \begin{forthel}
    \begin{lemma} % Satz 3.1
      $x-y-y$.
    \end{lemma}
  \end{forthel}


  \paragraph{Symmetry of betweenness}

  \begin{forthel}
    \begin{lemma}[SymmetryBetweenness] % Satz 3.2
      Assume $x-y-z$. Then $z-y-x$.
    \end{lemma}
  \end{forthel}


  \paragraph{Left-betweenness}

  \begin{forthel}
    \begin{lemma}[L3o3]
      $x-x-y$.
    \end{lemma}

    \begin{lemma} % Satz 3.4
      Assume $\Betw{x}{y}{z}$ and $\Betw{y}{x}{z}$.
      Then $x = y$.
    \end{lemma}
    \begin{proof}
      Take a point $w$ such that
      $\Betw{y}{w}{y}$ and $\Betw{x}{w}{x}$.
      Then $x = w = y$.
    \end{proof}

    \begin{lemma} % Satz 3.5 (first part)
      Assume $x-y-v$ and $y-z-v$. Then $x-y-z$.
    \end{lemma}
    \begin{proof}
      Take a point $w$ such that
      $\Betw{y}{w}{y}$ and $\Betw{z}{w}{x}$.
    \end{proof}

    \begin{lemma}[L3o7a]
      Assume $x-y-z$ and $y-z-r$ and $y \neq z$. Then $x-z-r$.
    \end{lemma}
    \begin{proof}
    	Take $v$ such that $x-z-v$ and $z-v : z-r$.	Then $y-z-v$ and $z-v : z-r$. Hence $v = r$.
    \end{proof}

    \begin{lemma}[S35b] % Satz 3.5 (second part)
      Assume $x-y-v$ and $y-z-v$. Then $x-z-v$.
    \end{lemma}
    \begin{proof}
      If $y = z$ then $\Betw{x}{z}{v}$.
      Assume $y \neq z$.
      We have $\Betw{x}{y}{z}$.
    \end{proof}

    \begin{lemma}[L3o6a]
      Assume $x-y-z$ and $x-z-r$. Then $y-z-r$.
    \end{lemma}

    \begin{lemma} % Satz 3.6 (second part)
      Assume $x-y-z$ and $x-z-r$. Then $x-y-r$.
    \end{lemma}
    \begin{proof}
      We have $\Betw{r}{z}{x}$.
      We have $\Betw{z}{y}{x}$.
      Thus $\Betw{r}{y}{x}$ (by S35b).
      Thus $\Betw{x}{y}{z}$.
      % We have $\Betw{z}{y}{r}$.
    \end{proof}



    \begin{lemma}[L3o7b]
      Assume $x-y-z$ and $y-z-r$ and $y \neq z$. Then $x-y-r$.
    \end{lemma}
  \end{forthel}

  Existence of at least two points follows from A8. (All other axioms also hold in a one-point space.)

  \begin{forthel}
    \begin{lemma}[L3o13]
      $x \neq y$ for some $x, y$.
    \end{lemma}

    \begin{lemma}[L3o14]
      There exist $z$ such that $x-y-z$ and $y \neq z$.
    \end{lemma}

    \begin{lemma}[L3o17]
      Assume $x-y-z$ and $u-v-z$ and $x-p-u$. Then there exist $q$ such that ($p-q-z$ and $y-q-v$).
    \end{lemma}
    \begin{proof}
      $x-p-u$ and $z-v-u$.
    	Take $r$ such that $v-r-x$ and $p-r-z$. % A7 (Pasch).
    	Take $q$ such that $r-q-z$ and $v-q-y$. % A7 (Pasch).
    \end{proof}
  \end{forthel}


  This is definition 4.1 in \textit{SST}.
  We say that the points $x,y,z,r,u,v,w,p$ are
  in an inner five-segment-configuration
  whenever $IFS(x,y,z,r,u,v,w,p)$.

  \begin{forthel}
    \begin{definition}[IFS]
      $IFS(x,y,z,r,v,w,p,q)$ iff ($x-y-z$ and $v-w-p$ and $x-z : v-p$ and $y-z : w-p$ and $x-r : v-q$ and $z-r : p-q$).
    \end{definition}
  \end{forthel}

  We can swap $x, y$ with $v, w$.

  \begin{forthel}
    \begin{axiom}[L4o2]
      Assume $IFS(x,y,z,r,v,w,p,q)$. Then $y-r : w-q$.
    \end{axiom}
  \end{forthel}

  If we have two three-point segments, with the same total length and each with a segment of the same length, then the remaining segments must also have the same length.

  %
  % TODO: Write a proof for this.
  %

  \begin{forthel}
    \begin{axiom}[L4o3]
      Assume $x-y-z$ and $r-v-w$ and $x-z : r-w$ and $y-z : v-w$. Then $x-y : r-v$.
    \end{axiom}

    \begin{definition}[L4o4]
      $x-y-z : u-v-w$ iff $x-y : u-v$ and $x-z : u-w$ and $y-z : v-w$.
    \end{definition}

    \begin{lemma}[L4o5]
      Assume $x-y-z$ and $x-z : r-w$. Then there exists $v$ such that ($r-v-w$ and $x-y-z : r-v-w$).
    \end{lemma}
    \begin{proof}
    	Take $u$ such that $w-r-u$ and $r \neq u$. Then take $v$ such that $u-r-v$ and $r-v : x-y$. Take $g$ such that $u-v-g$ and $v-g : y-z$. Then $x-z : r-w$. Therefore$ g = w$.
    \end{proof}

    \begin{lemma}[L4o6]
      Assume $x-y-z$ and $x-y-z : r-v-w$. Then $r-v-w$.
    \end{lemma}
    \begin{proof}
    	Take $u$ such that $r-u-w$ and $x-y-z : r-u-w$.	Then $r-u-w : r-v-w$ and $IFS(r,u,w,u,r,u,w,v)$.	Then $u-u : u-v$. Hence $u = v$. Hence $r-v-w$.
    \end{proof}

    \begin{lemma}[L4o11a]
      Assume $\Col{x}{y}{z}$. Then $\Col{y}{z}{x}$.
    \end{lemma}

    \begin{lemma}[L4o11b]
      Assume $\Col{x}{y}{z}$. Then $\Col{z}{x}{y}$.
    \end{lemma}

    \begin{lemma}[L4o11c]
      Assume $\Col{x}{y}{z}$. Then $\Col{z}{y}{x}$.
    \end{lemma}

    \begin{lemma}[L4o11d]
      Assume $\Col{x}{y}{z}$. Then $\Col{y}{x}{z}$.
    \end{lemma}

    \begin{lemma}[L4o11e]
      Assume $\Col{x}{y}{z}$. Then $\Col{x}{z}{y}$.
    \end{lemma}

    \begin{lemma}[L4o12]
      $\Col{x}{x}{y}$.
    \end{lemma}

    \begin{lemma}[L4o13]
      Assume $\Col{x}{y}{z}$ and $x-z : r-w$ and $r-v-w$. Then $\Col{r}{v}{w}$.
    \end{lemma}

    \begin{lemma}[L4o14o1]
      $x-y-z : u-v-w$ iff $y-x-z : v-u-w$.
    \end{lemma}

    \begin{lemma}[L4o14o2]
      $x-y-z : u-v-w$ iff $z-y-x : w-v-u$.
    \end{lemma}

    \begin{lemma}[L4o14o3]
      $x-y-z : u-v-w$ iff $x-z-y : u-w-v$.
    \end{lemma}

    \begin{axiom}[L4o14]
      Assume $\Col{x}{y}{z}$ and $x-y : r-v$. Then there exists $w$ such that $x-y-z : r-v-w$.
    \end{axiom}

    \begin{definition}[L4o15]
      $FS(x,y,z,r,v,w,p,q)$ iff $\Col{x}{y}{z}$ and $x-y-z : v-w-p$ and $x-r : v-q$ and $y-r : w-q$.
    \end{definition}

    \begin{axiom}[L4o16]
      Assume $FS(x,y,z,r,v,w,p,q)$ and $x \neq y$. Then $z-r : p-q$.
    \end{axiom}


    \begin{lemma}[L4o17]
      Assume $x \neq y$ and $\Col{x}{y}{z}$ and $x-p : x-q$ and $y-p : y-q$. Then $z-p : z-q$.
    \end{lemma}
    \begin{proof}
    	$FS(x,y,z,p,x,y,z,q)$.
    \end{proof}


    \begin{lemma}[L4o18]
      Assume $x \neq y$ and $\Col{x}{y}{z}$ and $x-z : x-p$ and $y-z : y-p$. Then $z = p$.
    \end{lemma}

    \begin{lemma}[L4o19]
      Assume $x-z-y$ and $x-z : x-p$ and $y-z : y-p$. Then $z = p$.
    \end{lemma}
    \begin{proof}
      Assume $x = y$. Then $x = z$ and $x = p$. Hence $z = p$. Assume $x \neq y$.
    \end{proof}
  \end{forthel}

  The 11th axiom of Tarski's axiomatic system says that if $x-y-w$ and $x-z-w$ then either $x-y-z$ or $x-z-y$. To show that it follows from the first ten axioms we first prove Lemma C5o1 from which we can easy deduce the 11th axiom.

  The definitions, lemmas and axioms C5o1a - C5o1p are not part of \textit{SST}. We have opted for adding them to improve proof-checking speed and readability of the text.

  \begin{forthel}
    \begin{definition}[C5o1a]
      $Betw5(x,y,z,r,p)$ iff $x-y-z$ and $x-y-r$ and $x-y-p$ and $x-z-r$ and $x-z-p$ and $x-r-p$ and $y-z-r$ and $y-z-p$ and $y-r-p$ and $z-r-p$.
    \end{definition}

    Let $x~y~z~r~p$ stand for $Betw5(x,y,z,r,p)$.
  \end{forthel}

  The following 4 predicates state the already proven statements for different positions in the proof.
  They are not defined in the book \textit{Metamathematische Methoden in der Geometrie}.
  We use them because they seem to increase the performance of the proof assistant,
  when checked just before the next proving step.

  \begin{forthel}
    \begin{definition}[C5o1b]
      $Th(x,y,z,r,p,q,g,h)$ iff $x \neq y$ and $x-y-z$ and $x-y-r$ and $x-r-p$ and $r-p : z-r$ and $x-z-q$ and $z-q : z-r$ and $z-q-h$ and $r-p-g$.
    \end{definition}

    \begin{definition}[C5o1c]
      $Th2(x,y,z,r,p,q,g)$ iff $x \neq y$ and $x~y~z~q~g$ and $x~y~r~p~g$ and $r-p : z-r$ and $z-q : z-r$ and $y-p : g-z$ and $y-g : g-y$.
    \end{definition}

    \begin{definition}[C5o1d]
      $Th3(x,y,z,r,p,q,g,u)$ iff $Th2(x,y,z,r,p,q,g)$ and $\OFS{y}{z}{q}{p}{g}{p}{r}{z}$ and $p-q : z-r$ and $z-u-p$ and $r-u-q$ and $IFS(r,u,q,z,r,u,q,p)$ and $IFS(z,u,p,r,z,u,p,q)$ and $u-r : u-q$.
    \end{definition}

    \begin{definition}[C5o1e]
      $Th4(x,y,z,r,p,q,g,u,v,w,h)$ iff $Th3(x,y,z,r,p,q,g,u)$ and $z \neq p$ and $z \neq q$ and $p-z-v$ and $z-v : z-q$ and $q-z-h$ and $z-h : z-u$ and $v-h-w$ and $h-w : h-v$.
    \end{definition}
  \end{forthel}

  For the following 5 Statements we did not find a proof yet that gets checked positive by Naproche SAD. They are all used in the proof of Lemma C5o1p and Lemma C5o1.

  \begin{forthel}
    \begin{lemma}[C5o1f]
      Assume $x \neq y$ and $x-y-z$ and $x-y-r$. Then there exist points $a,b$ such that $x-r-a$ and $r-a : z-r$ and $x-z-b$ and $z-b : z-r$.
    \end{lemma}
    \begin{proof}
    	Take point $a$ such that $x-r-a$ and $r-a : z-r$ (by A4).	Take point $b$ such that $x-z-b$ and $z-b : z-r$ (by A4).
    \end{proof}

    \begin{axiom}[C5o1g]
      If $Th(x,y,z,r,p,q,g,h)$ then $x~y~z~q~h$ and $x~y~r~p~g$.
    \end{axiom}

    \begin{axiom}[C5o1h]
      Assume $Th(x,y,z,r,p,q,g,h)$ and $x~y~z~q~h$ and $x~y~r~p~g$. Then $y-p : h-z$.
    \end{axiom}

    \begin{axiom}[C5o1i]
      Assume $Th2(x,y,z,r,p,q,g)$ and $\OFS{y}{z}{q}{p}{g}{p}{r}{z}$. Then $p-q : z-r$.
    \end{axiom}

    \begin{axiom}[C5o1j]
      Assume $Th(x,y,z,r,p,q,g,h)$ and $x~y~z~q~h$ and $x~y~r~p~g$ and $y-p : h-z$. Then $y-g : h-y$.
    \end{axiom}



    \begin{lemma}[C5o1k]
      Assume $x \neq y$ and $x-y-z$ and $x-y-r$ and $x-r-p$ and $r-p : z-r$ and $x-z-q$ and $z-q : z-r$ and ($z = p$ or $r = q$). Then $x-z-r$ or $x-r-z$.
    \end{lemma}

    \begin{lemma}[C5o1l]
      Assume $x \neq y$ and $x-y-z$ and $x-y-r$ and $x-r-p$ and $r-p : z-r$ and $x-z-q$ and $z-q : z-r$. Then there exist points $s,t$ such that $z-q-t$ and $r-p-s$.
    \end{lemma}

    \begin{lemma}[C5o1m]
      Assume $Th2(x,y,z,r,p,q,g)$. Then $\OFS{y}{z}{q}{p}{g}{p}{r}{z}$.
    \end{lemma}

    \begin{lemma}[C5o1n]
      Assume $Th2(x,y,z,r,p,q,g)$ and $\OFS{y}{z}{q}{p}{g}{p}{r}{z}$ and $p-q : z-r$. Then there exist $u$ such that $z-u-p$ and $r-u-q$.
    \end{lemma}

    \begin{lemma}[C5o1o]
      Assume $Th4(x,y,z,r,p,q,g,u,v,w,h)$. Then $\OFS{q}{z}{h}{v}{v}{z}{u}{q}$ and $h-v : u-q$ and $h-w : u-r$ and $\OFS{q}{u}{r}{z}{v}{h}{w}{z}$ and $q-r : v-w$ and $z-w : z-r$ and $z-v : z-w$.
    \end{lemma}
    \begin{proof}
    	$\OFS{q}{z}{h}{v}{v}{z}{u}{q}$. Hence $h-v : u-q$.
      Hence $h-w : u-r$. % TODO: this step is slow.
    	Therefore $\OFS{q}{u}{r}{z}{v}{h}{w}{z}$. Hence $q-r : v-w$. If $q \neq u$ then $z-w : z-r$.
      If $q = u$ then $q = r$. % TODO: this step is slow.
    	Then $v = w$. Therefore $z-w : z-r$. Hence $z-v : z-w$.
    \end{proof}
  \end{forthel}

  The Idea to proof Lemma C5o1 is to extend the line $x-z$ and $x-r$ through two points $p,q$ such that $r-p : z-r$ and $z-q : z-r$. Then one can easy see that if either $z = p$ or $r = q$, $x-y-z$ or $x-z-y$ must hold. The following Lemma proofs that if there exist such points $p$ and $q$ then $z = p$ or $r = q$ must hold. To see that such points exist one has to use Axiom A4 twice.

  \begin{forthel}
    \begin{lemma}[C5o1p]
      Assume $x \neq y$ and $x-y-z$ and $x-y-r$ and $x-r-p$ and $r-p : z-r$ and $x-z-q$ and $z-q : z-r$. Then $z = p$ or $r = q$.
    \end{lemma}
    \begin{proof}
    	Take points $s,t$ such that $z-q-t$ and $r-p-s$.
    	Then $Th(x,y,z,r,p,q,s,t)$.
    	Then $x~y~z~q~t$ and $x~y~r~p~s$.
    	Then $y-p : t-z$.
    	Then $y-t : t-y$.
    	Then $s = t$.
    	Then $Th2(x,y,z,r,p,q,s)$.
    	Then $\OFS{y}{z}{q}{p}{s}{p}{r}{z}$.
    	Then $p-q : z-r$.
    	Take $u$ such that $z-u-p$ and $r-u-q$ (by C5o1n).
    	Then $IFS(r,u,q,z,r,u,q,p)$.
    	Then $IFS(z,u,p,r,z,u,p,q)$.
    	Then $u-z : u-p$.
    	Then $u-r : u-q$.
    	Assume $z \neq p$. Then $z \neq q$.
    		Take $v$ such that $p-z-v$ and $z-v : z-q$.
    		Take $h$ such that $q-z-h$ and $z-h : z-u$.
    		Take $w$ such that $v-h-w$ and $h-w : h-v$.
        Then $\OFS{q}{z}{h}{v}{v}{z}{u}{q}$
        and $h-v : u-q$ and $h-w : u-r$
        and $\OFS{q}{u}{r}{z}{v}{h}{w}{z}$
        and $q-r : v-w$ and $z-w : z-r$ and $z-v : z-w$ (by C5o1o).
    		Then $h-v : h-w$. $z \neq p$. Hence $h \neq z$. $\Col{h}{z}{q}$. Therefore $q-v : q-w$.
    		Then $z-v : z-w$. $z \neq q$. $\Col{z}{q}{y}$. Therefore $y-v : y-w$.
    		Then $z-v : z-w$. $z \neq q$. $\Col{z}{q}{s}$. Therefore $s-v : s-w$.
    		Then $y \neq s$.
    		Then $q = r$.
    	Assume $z = p$.
    \end{proof}

    \begin{lemma}[D5o1]
      Assume $x \neq y$ and $x-y-z$ and $x-y-r$. Then $x-z-r$ or $x-r-z$.
    \end{lemma}
    \begin{proof}
    	Take $p,q$ such that $x-r-p$ and $r-p : z-r$ and $x-z-q$ and $z-q : z-r$. Then $z = p$ or $r = q$ (by C5o1p). Therefore $x-z-r$ or $x-r-z$ (by C5o1k).
    \end{proof}

    \begin{lemma}[D5o2]
      Assume $x \neq y$ and $x-y-z$ and $x-y-r$. Then $y-z-r$ or $y-r-z$.
    \end{lemma}

    \begin{theorem}[D5o3]
      If $x-y-w$ and $x-z-w$ then $x-y-z$ or $x-z-y$.
    \end{theorem}
  \end{forthel}

\end{document}
